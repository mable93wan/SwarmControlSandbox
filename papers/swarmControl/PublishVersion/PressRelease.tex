\documentclass{article}
%\documentclass[letterpaper, 10 pt, conference]{ieeeconf}
\usepackage{xcolor}
\newcommand\todo[1]{\textcolor{red}{#1}}   %I use this command to highlight areas that are unfinished
%\renewcommand\todo[1]{}   % uncomment this to hide all \todo{} in the text.
\usepackage{hyperref}
\usepackage{geometry}
\usepackage{overpic}
\usepackage{wrapfig}
\graphicspath{{./pictures/pdf/},{./pictures/ps/},{./pictures/png/},{./pictures/jpg/}}
\begin{document}
\author{Shiva Shahrokhi and  Aaron T. Becker\\ University of Houston, Houston, TX 77204-4005 USA\\ {\tt\small  sshahrokhi2@uh.edu, atbecker@uh.edu}}
\title{Stochastic Swarm Control with Global Inputs}
\date{}
\pagenumbering{gobble}
\maketitle
\begin{wrapfigure}{r}{7cm}
\centering
\begin{overpic}[width=7cm]{BlockPushing1.png}\end{overpic}
\caption{\label{fig:bigPictureMeanAndVarianceForSwarm} A swarm of robots, all controlled by a uniform force field, can be effectively controlled by a hybrid controller that knows only the mean and variance of the swarm's position.  Here a swarm of simple robots (blue discs) pushes a black block toward the goal. See video at \href{https://youtu.be/tCej-9e6-4o}{https://youtu.be/tCej-9e6-4o}.}
\end{wrapfigure}

Micro- and nanorobots can be built in large numbers, but controlling each robot individually is prohibitively difficult. Instead, micro- and nanorobots are often controlled by a global field, and each robot receives the same control inputs.  For example ``everyone move up" or ``everyone move left". In previous work we conducted large-scale human-user experiments where humans played games that steered large swarms of simple robots to complete tasks such as manipulating blocks using the site \href{www.swarmcontrol.net}{\emph{SwarmControl.net}}. One surprising result was that humans completed a block-pushing task faster when provided with only the mean and variance of the robot swarm than with the position of every robot. Inspired by human operators, this paper investigates automatic controllers that use only the mean and variance of a robot swarm's position. We prove that the mean position is controllable, and show how obstacles can make the variance controllable. We next derive automatic controllers for these and a hybrid, hysteresis-based switching control to regulate the mean and variance of the swarm's position. To demonstrate the utility of these controllers, we use them in an algorithm that automatically steers a swarm of robots to push a block through a maze and into a goal region.





\end{document}

 