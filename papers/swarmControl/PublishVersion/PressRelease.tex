\documentclass[letterpaper, 10 pt, conference]{ieeeconf}
\usepackage{xcolor}
\newcommand\todo[1]{\textcolor{red}{#1}}   %I use this command to highlight areas that are unfinished
%\renewcommand\todo[1]{}   % uncomment this to hide all \todo{} in the text.
\usepackage{hyperref}
\usepackage{geometry}
\usepackage{overpic}
\graphicspath{{./pictures/pdf/},{./pictures/ps/},{./pictures/png/},{./pictures/jpg/}}
\begin{document}
\author{}
\title{Stochastic Swarm Control with Global Inputs}
\maketitle


Micro- and nanorobots can be built in large numbers, but controlling each robot individually is prohibitively difficult. Instead, micro- and nanorobots are often controlled by a global field. In previous work we conducted large-scale human-user experiments where humans played games that steered large swarms of simple robots to complete tasks such as manipulating blocks. One surprising result was that humans completed a block-pushing task faster when provided with only the mean and variance of the robot swarm than with full-state feedback. Inspired by human operators, this paper investigates controllers that use only the mean and variance of a robot swarm. We prove that the mean position is controllable, and show how an obstacle can make the variance controllable. We next derive automatic controllers for these and a hybrid, hysteresis-based switching control to regulate the first two moments of the robot distribution. Finally, we employ these controllers as primitives for a block-pushing task. 

\begin{figure}
\centering
\begin{overpic}[width=0.9\columnwidth]{BlockPushing1.png}\end{overpic}
%\todo{I like the 'target' symbol, but it is not self-documenting.  We need a legend explaining the min and max variance ellipses, the goal region, the variance, the mean, the object COM, and the target mean position.  I think these are easiest to make in powerpoint.
%Please use the same color and line style for the variance min and max as you use in Figure 4.
%}
%{blockpushingImageWithMeanAndVarianceOverlay.png}
\caption{\label{fig:bigPictureMeanAndVarianceForSwarm} A swarm of robots, all controlled by a uniform force field, can be effectively controlled by a hybrid controller that knows only the first and second moments of the robot distribution.  Here a swarm of simple robots (blue discs) pushes a black block toward the goal.}
\end{figure}



\end{document}

 