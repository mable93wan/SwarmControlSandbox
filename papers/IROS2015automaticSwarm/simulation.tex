
\section{Simulation}\label{sec:simulation}
% Simulation:
%%  Mean control results   (image and plot(s) )
%%  Variance Control      (image and plot(s) )
%%  Hybrid control   (image and plot(s) )

\subsection{Controlling the mean position}

For controlling mean position, we use a PD controller that uses the mean position and mean velocity. Our control input is the force applied to each robot, and goal positions are the desired positions:
\begin{align}
u_x &= K_{p}(x_{goal} - \bar{x}) + K_{d}(0-\bar{v}_x) \nonumber\\
u_y &= K_{p}(y_{goal}  - \bar{y}) + K_{d}(0-\bar{v}_y)  \label{eq:PDcontrolPosition}
\end{align}
here $K_{p}$ is the proportional gain, and $K_{d}$ is the derivative gain. We performed a parameter sweep to identify the best values.  Representative experiments are shown in Fig.~\ref{fig:gainvalues}. 100 robots were used and the maximum speed was 3 meter per second. As shown in Fig.~\ref{fig:gainvalues}, we can achieve an overshoot of 1\% and a  rise time of 1.52 s with $K_{p}= 4$, and  $K_{d} = 1$. 

\begin{figure}
\centering
\begin{overpic}[width = \columnwidth]{gainvalues.eps}
\end{overpic}
\vspace{-1em}
\caption{\label{fig:gainvalues} Tuning proportional ($K_p$, top) and derivative ($K_d$, bottom)  gain values in \eqref{eq:PDcontrolPosition} improves performance. These plots show convergence with 100 robots.
%\vspace{-2em}
}
\end{figure}



%give PID control law, explain experiment (number of robots, maximum speed, ).

%contrast controllers -- as is typical with PID control laws, we can tune the response to meet desired specifications.


%image showing varying P control  %I want  1.5 cycles, nicely cropped,  all starting at same time

%image showing varying D control

%image showing varying the number of robots n % is this needed?

%\subfloat[][Vary Visual Feedback]{\label{fig:VaryVis}
%\begin{overpic}[width =\figwid]{VaryVisFS.pdf}\end{overpic}}


%\begin{figure}
%        \centering
%        \begin{subfigure}[b]{0.3\textwidth}
%                \includegraphics[width=\textwidth]{fig/gain1d1.pdf}
%                \caption{g 1, d 1}
%                \label{fig:gull}
%        \end{subfigure}%
%        ~ %add desired spacing between images, e. g. ~, \quad, \qquad, \hfill etc.
%          %(or a blank line to force the subfigure onto a new line)
%        \begin{subfigure}[b]{0.3\textwidth}
%                \includegraphics[width=\textwidth]{fig/gain2d1.pdf}
%                \caption{g 2, d 1}
%                \label{fig:tiger}
%        \end{subfigure}
%        ~ %add desired spacing between images, e. g. ~, \quad, \qquad, \hfill etc.
%          %(or a blank line to force the subfigure onto a new line)
%        \begin{subfigure}[b]{0.3\textwidth}
%                \includegraphics[width=\textwidth]{fig/gain4d1.pdf}
%                \caption{g 4, d 1}
%                \label{fig:mouse}
%        \end{subfigure}
%                \begin{subfigure}[b]{0.3\textwidth}
%                \includegraphics[width=\textwidth]{fig/gain5d1.pdf}
%                \caption{g 5, d 1}
%                \label{fig:gull}
%        \end{subfigure}%
%        ~ %add desired spacing between images, e. g. ~, \quad, \qquad, \hfill etc.
%          %(or a blank line to force the subfigure onto a new line)
%        \begin{subfigure}[b]{0.3\textwidth}
%                \includegraphics[width=\textwidth]{fig/gain8d1.pdf}
%                \caption{g 8, d 1}
%                \label{fig:tiger}
%        \end{subfigure}
%        ~ %add desired spacing between images, e. g. ~, \quad, \qquad, \hfill etc.
%          %(or a blank line to force the subfigure onto a new line)
%        \begin{subfigure}[b]{0.3\textwidth}
%                \includegraphics[width=\textwidth]{fig/gain100d1.pdf}
%                \caption{g 100, d 1}
%                \label{fig:mouse}
%        \end{subfigure}
%        \caption{Different Gain Values}\label{fig:gainvalues}
%\end{figure}
%\begin{figure}
%        \centering
%        \begin{subfigure}[b]{0.3\textwidth}
%                \includegraphics[width=\textwidth]{fig/gain4d2.pdf}
%                \caption{g 4, d 2}
%                \label{fig:gull}
%        \end{subfigure}%
%        ~ %add desired spacing between images, e. g. ~, \quad, \qquad, \hfill etc.
%          %(or a blank line to force the subfigure onto a new line)
%        \begin{subfigure}[b]{0.3\textwidth}
%                \includegraphics[width=\textwidth]{fig/gain4d4.pdf}
%                \caption{g 4, d 4}
%                \label{fig:tiger}
%        \end{subfigure}
%        ~ %add desired spacing between images, e. g. ~, \quad, \qquad, \hfill etc.
%          %(or a blank line to force the subfigure onto a new line)
%        \begin{subfigure}[b]{0.3\textwidth}
%                \includegraphics[width=\textwidth]{fig/gain4d05.pdf}
%                \caption{g 4, d 0.5}
%                \label{fig:mouse}
%        \end{subfigure}
%         \caption{Different Derivative Values}\label{fig:animals}
%\end{figure}


\subsection{Controlling the variance}
\begin{figure}
\centering
\begin{overpic}[width = \columnwidth] {brownianfig.eps}
\end{overpic}
\vspace{-1em}
\caption{\label{fig:varyBrownian} Increased noise results in more responsive variance control because stronger Brownian noise causes a faster increase of variance.
%\vspace{-2em}
}
\end{figure}

%cite the control law, explain experiment (number of robots, maximum speed, ).

For variance control we use the control law discussed in Section~\ref{sec:VarianceControl}.  Moving away from the wall and waiting is sufficient to increase variance because Brownian noise naturally disperses the swarm in such a way that the variance increases linearly~\cite{einstein1956investigations}.  If faster dispersion is needed, the swarm can be pushed through obstacles such as a diffraction grating or Pachinko board~\cite{Becker2013b}. To decrease the variance, we push the swarm into corners. Our algorithm identifies the nearest corner by calculating distance to the corners. Corners are sorted. If the block is ahead of the swarm, the swarm will go to the previous corner for not pushing the block backward. \todo{ Is this explanation good?}
The variance control law has three gains:
\begin{align}
u_x &= K_{p}(x_{goal}(\sigma^2_{ref}) - \bar{x}) - K_{d}\bar{v}_x + K_{i}(\sigma^2_{ref}-\sigma^2_{x}) \nonumber\\
u_y &= K_{p}(y_{goal}(\sigma^2_{ref})  - \bar{y}) - K_{d}\bar{v}_y + K_{i}(\sigma^2_{ref}-\sigma^2_{y}).  \label{eq:PDcontrolVariance}
\end{align}
The following assumes that the nearest wall is to the left of the robot at $x=0$:
\begin{align}
x_{goal}(\sigma^2_{ref}) = r + \sqrt{3\sigma^2_{ref}}
\end{align}
 If another wall is closer, the signs of $K_p,K_i$ are inverted, and the location $x_{goal}$ is translated. In a slight abuse of notation, because the variance integrates over time, we call the gain scaling the variance error $K_i$.  Results are shown in Fig.~\ref{fig:varyBrownian}, with $K_{p,i,d} = [4,1,1]$.

%\todo{Are the gain values and control law correct?}s

\todo{image showing control x variance and y-variance out of phase}


\subsection{Hybrid Control of mean and variance}

Figure~\ref{fig:hybrid} shows representative results of the hybrid controller given in Alg.~\ref{alg:MeanVarianceControl} with 100 robots in a square workspace with no internal obstacles.

%\todo{plot showing 1.5 cycles of mean position, and a variance goal.  We might need a longer time}
\begin{figure}
\centering
\begin{overpic}[scale=0.35]{MeanVariance2.eps}
\end{overpic}
\vspace{-2em}
\caption{\label{fig:hybrid} Simulation result with 100 robots under hybrid control Alg.~\ref{alg:MeanVarianceControl}, which  controls both the mean position (top) and variance (bottom). For ease of analysis, only $x$ position and variance are shown.
%\vspace{-2em}
}
\end{figure}






